\PassOptionsToPackage{unicode=true}{hyperref} % options for packages loaded elsewhere
\PassOptionsToPackage{hyphens}{url}
%
\documentclass[]{article}
\usepackage{lmodern}
\usepackage{amssymb,amsmath}
\usepackage{ifxetex,ifluatex}
\usepackage{fixltx2e} % provides \textsubscript
\ifnum 0\ifxetex 1\fi\ifluatex 1\fi=0 % if pdftex
  \usepackage[T1]{fontenc}
  \usepackage[utf8]{inputenc}
  \usepackage{textcomp} % provides euro and other symbols
\else % if luatex or xelatex
  \usepackage{unicode-math}
  \defaultfontfeatures{Ligatures=TeX,Scale=MatchLowercase}
\fi
% use upquote if available, for straight quotes in verbatim environments
\IfFileExists{upquote.sty}{\usepackage{upquote}}{}
% use microtype if available
\IfFileExists{microtype.sty}{%
\usepackage[]{microtype}
\UseMicrotypeSet[protrusion]{basicmath} % disable protrusion for tt fonts
}{}
\IfFileExists{parskip.sty}{%
\usepackage{parskip}
}{% else
\setlength{\parindent}{0pt}
\setlength{\parskip}{6pt plus 2pt minus 1pt}
}
\usepackage{hyperref}
\hypersetup{
            pdftitle={Regression Models: Assignment 1},
            pdfauthor={Daniel Alonso},
            pdfborder={0 0 0},
            breaklinks=true}
\urlstyle{same}  % don't use monospace font for urls
\usepackage[margin=1in]{geometry}
\usepackage{color}
\usepackage{fancyvrb}
\newcommand{\VerbBar}{|}
\newcommand{\VERB}{\Verb[commandchars=\\\{\}]}
\DefineVerbatimEnvironment{Highlighting}{Verbatim}{commandchars=\\\{\}}
% Add ',fontsize=\small' for more characters per line
\usepackage{framed}
\definecolor{shadecolor}{RGB}{248,248,248}
\newenvironment{Shaded}{\begin{snugshade}}{\end{snugshade}}
\newcommand{\AlertTok}[1]{\textcolor[rgb]{0.94,0.16,0.16}{#1}}
\newcommand{\AnnotationTok}[1]{\textcolor[rgb]{0.56,0.35,0.01}{\textbf{\textit{#1}}}}
\newcommand{\AttributeTok}[1]{\textcolor[rgb]{0.77,0.63,0.00}{#1}}
\newcommand{\BaseNTok}[1]{\textcolor[rgb]{0.00,0.00,0.81}{#1}}
\newcommand{\BuiltInTok}[1]{#1}
\newcommand{\CharTok}[1]{\textcolor[rgb]{0.31,0.60,0.02}{#1}}
\newcommand{\CommentTok}[1]{\textcolor[rgb]{0.56,0.35,0.01}{\textit{#1}}}
\newcommand{\CommentVarTok}[1]{\textcolor[rgb]{0.56,0.35,0.01}{\textbf{\textit{#1}}}}
\newcommand{\ConstantTok}[1]{\textcolor[rgb]{0.00,0.00,0.00}{#1}}
\newcommand{\ControlFlowTok}[1]{\textcolor[rgb]{0.13,0.29,0.53}{\textbf{#1}}}
\newcommand{\DataTypeTok}[1]{\textcolor[rgb]{0.13,0.29,0.53}{#1}}
\newcommand{\DecValTok}[1]{\textcolor[rgb]{0.00,0.00,0.81}{#1}}
\newcommand{\DocumentationTok}[1]{\textcolor[rgb]{0.56,0.35,0.01}{\textbf{\textit{#1}}}}
\newcommand{\ErrorTok}[1]{\textcolor[rgb]{0.64,0.00,0.00}{\textbf{#1}}}
\newcommand{\ExtensionTok}[1]{#1}
\newcommand{\FloatTok}[1]{\textcolor[rgb]{0.00,0.00,0.81}{#1}}
\newcommand{\FunctionTok}[1]{\textcolor[rgb]{0.00,0.00,0.00}{#1}}
\newcommand{\ImportTok}[1]{#1}
\newcommand{\InformationTok}[1]{\textcolor[rgb]{0.56,0.35,0.01}{\textbf{\textit{#1}}}}
\newcommand{\KeywordTok}[1]{\textcolor[rgb]{0.13,0.29,0.53}{\textbf{#1}}}
\newcommand{\NormalTok}[1]{#1}
\newcommand{\OperatorTok}[1]{\textcolor[rgb]{0.81,0.36,0.00}{\textbf{#1}}}
\newcommand{\OtherTok}[1]{\textcolor[rgb]{0.56,0.35,0.01}{#1}}
\newcommand{\PreprocessorTok}[1]{\textcolor[rgb]{0.56,0.35,0.01}{\textit{#1}}}
\newcommand{\RegionMarkerTok}[1]{#1}
\newcommand{\SpecialCharTok}[1]{\textcolor[rgb]{0.00,0.00,0.00}{#1}}
\newcommand{\SpecialStringTok}[1]{\textcolor[rgb]{0.31,0.60,0.02}{#1}}
\newcommand{\StringTok}[1]{\textcolor[rgb]{0.31,0.60,0.02}{#1}}
\newcommand{\VariableTok}[1]{\textcolor[rgb]{0.00,0.00,0.00}{#1}}
\newcommand{\VerbatimStringTok}[1]{\textcolor[rgb]{0.31,0.60,0.02}{#1}}
\newcommand{\WarningTok}[1]{\textcolor[rgb]{0.56,0.35,0.01}{\textbf{\textit{#1}}}}
\usepackage{graphicx,grffile}
\makeatletter
\def\maxwidth{\ifdim\Gin@nat@width>\linewidth\linewidth\else\Gin@nat@width\fi}
\def\maxheight{\ifdim\Gin@nat@height>\textheight\textheight\else\Gin@nat@height\fi}
\makeatother
% Scale images if necessary, so that they will not overflow the page
% margins by default, and it is still possible to overwrite the defaults
% using explicit options in \includegraphics[width, height, ...]{}
\setkeys{Gin}{width=\maxwidth,height=\maxheight,keepaspectratio}
\setlength{\emergencystretch}{3em}  % prevent overfull lines
\providecommand{\tightlist}{%
  \setlength{\itemsep}{0pt}\setlength{\parskip}{0pt}}
\setcounter{secnumdepth}{0}
% Redefines (sub)paragraphs to behave more like sections
\ifx\paragraph\undefined\else
\let\oldparagraph\paragraph
\renewcommand{\paragraph}[1]{\oldparagraph{#1}\mbox{}}
\fi
\ifx\subparagraph\undefined\else
\let\oldsubparagraph\subparagraph
\renewcommand{\subparagraph}[1]{\oldsubparagraph{#1}\mbox{}}
\fi

% set default figure placement to htbp
\makeatletter
\def\fps@figure{htbp}
\makeatother


\title{Regression Models: Assignment 1}
\author{Daniel Alonso}
\date{November 24th, 2020}

\begin{document}
\maketitle

Importing libraries

\begin{Shaded}
\begin{Highlighting}[]
\KeywordTok{library}\NormalTok{(dplyr)}
\end{Highlighting}
\end{Shaded}

\begin{verbatim}
## 
## Attaching package: 'dplyr'
\end{verbatim}

\begin{verbatim}
## The following objects are masked from 'package:stats':
## 
##     filter, lag
\end{verbatim}

\begin{verbatim}
## The following objects are masked from 'package:base':
## 
##     intersect, setdiff, setequal, union
\end{verbatim}

\begin{Shaded}
\begin{Highlighting}[]
\KeywordTok{library}\NormalTok{(MuMIn)}
\end{Highlighting}
\end{Shaded}

\hypertarget{exercise-1}{%
\section{Exercise 1}\label{exercise-1}}

\hypertarget{simulation}{%
\subsection{Simulation}\label{simulation}}

\begin{Shaded}
\begin{Highlighting}[]
\NormalTok{sim =}\StringTok{ }\KeywordTok{list}\NormalTok{()}
\ControlFlowTok{for}\NormalTok{ (j }\ControlFlowTok{in} \DecValTok{1}\OperatorTok{:}\DecValTok{1000}\NormalTok{) \{}
\NormalTok{    vals =}\StringTok{ }\KeywordTok{c}\NormalTok{()}
    \ControlFlowTok{for}\NormalTok{ (i }\ControlFlowTok{in} \DecValTok{1}\OperatorTok{:}\DecValTok{100}\NormalTok{) \{}
\NormalTok{        run =}\StringTok{ }\DecValTok{3} \OperatorTok{+}\StringTok{ }\DecValTok{3}\OperatorTok{*}\KeywordTok{cos}\NormalTok{(i}\OperatorTok{/}\DecValTok{10} \OperatorTok{+}\StringTok{ }\DecValTok{50}\NormalTok{) }\OperatorTok{+}\StringTok{ }\KeywordTok{rnorm}\NormalTok{(}\DecValTok{1}\NormalTok{, }\DataTypeTok{mean=}\DecValTok{0}\NormalTok{, }\DataTypeTok{sd=}\DecValTok{1}\NormalTok{)}
\NormalTok{        vals =}\StringTok{ }\KeywordTok{c}\NormalTok{(vals, run)}
\NormalTok{    \}}
\NormalTok{    sim[[j]] =}\StringTok{ }\NormalTok{vals}
\NormalTok{\}}
\NormalTok{sim}
\end{Highlighting}
\end{Shaded}

\hypertarget{exercise-2}{%
\section{Exercise 2}\label{exercise-2}}

\hypertarget{importing-the-data}{%
\subsection{Importing the data}\label{importing-the-data}}

\begin{Shaded}
\begin{Highlighting}[]
\NormalTok{d <-}\StringTok{ }\KeywordTok{data.frame}\NormalTok{(}\KeywordTok{read.table}\NormalTok{(}\StringTok{'../data/index.txt'}\NormalTok{, }\DataTypeTok{header=}\OtherTok{TRUE}\NormalTok{))}
\end{Highlighting}
\end{Shaded}

\begin{Shaded}
\begin{Highlighting}[]
\NormalTok{X =}\StringTok{ }\NormalTok{d}\OperatorTok{$}\NormalTok{PovPct}
\NormalTok{Y =}\StringTok{ }\NormalTok{d}\OperatorTok{$}\NormalTok{Brth15to17}
\NormalTok{beta1 =}\StringTok{ }\KeywordTok{cov}\NormalTok{(X, Y)}\OperatorTok{/}\KeywordTok{var}\NormalTok{(X)}
\NormalTok{beta0 =}\StringTok{ }\KeywordTok{mean}\NormalTok{(Y) }\OperatorTok{-}\StringTok{ }\NormalTok{beta1}\OperatorTok{*}\KeywordTok{mean}\NormalTok{(X)}
\NormalTok{beta1}
\end{Highlighting}
\end{Shaded}

\begin{verbatim}
## [1] 1.373345
\end{verbatim}

\begin{Shaded}
\begin{Highlighting}[]
\NormalTok{beta0}
\end{Highlighting}
\end{Shaded}

\begin{verbatim}
## [1] 4.267293
\end{verbatim}

\hypertarget{exercise-3}{%
\section{Exercise 3}\label{exercise-3}}

First we have the log-likelihood function for \(\beta\) and
\(\sigma^{2}\)

\(l(\sigma^{2} | X) = \sum_{i=1}^n log(\frac{1}{\sqrt{2 \pi \sigma^{2}}} - \frac{(Y_{i} - (\beta_{0} + \beta_{1} x_{ik} + \dots + \beta_{k} x_{ik}))^{2}}{2 \sigma^{2}})\)

\hypertarget{exercise-4}{%
\section{Exercise 4}\label{exercise-4}}

\hypertarget{exercise-5}{%
\section{Exercise 5}\label{exercise-5}}

\begin{Shaded}
\begin{Highlighting}[]
\NormalTok{bodyfat <-}\StringTok{ }\KeywordTok{data.frame}\NormalTok{(}\KeywordTok{read.table}\NormalTok{(}\StringTok{'../data/bodyfat.txt'}\NormalTok{, }\DataTypeTok{header=}\OtherTok{TRUE}\NormalTok{))}
\NormalTok{modall <-}\StringTok{ }\KeywordTok{lm}\NormalTok{(hwfat }\OperatorTok{~}\NormalTok{., }\DataTypeTok{data =}\NormalTok{ bodyfat)}
\KeywordTok{summary}\NormalTok{(modall)}
\end{Highlighting}
\end{Shaded}

\begin{verbatim}
## 
## Call:
## lm(formula = hwfat ~ ., data = bodyfat)
## 
## Residuals:
##    Min     1Q Median     3Q    Max 
## -6.162 -1.858 -0.464  2.502  8.177 
## 
## Coefficients:
##             Estimate Std. Error t value Pr(>|t|)    
## (Intercept) 13.29370    9.63027   1.380   0.1718    
## age         -0.32893    0.32158  -1.023   0.3098    
## ht          -0.06731    0.16051  -0.419   0.6762    
## wt          -0.01365    0.02591  -0.527   0.5999    
## abs          0.37142    0.08837   4.203 7.55e-05 ***
## triceps      0.38743    0.13761   2.815   0.0063 ** 
## subscap      0.11405    0.14193   0.804   0.4243    
## ---
## Signif. codes:  0 '***' 0.001 '**' 0.01 '*' 0.05 '.' 0.1 ' ' 1
## 
## Residual standard error: 3.028 on 71 degrees of freedom
## Multiple R-squared:  0.8918, Adjusted R-squared:  0.8827 
## F-statistic: 97.54 on 6 and 71 DF,  p-value: < 2.2e-16
\end{verbatim}

The sum of residuals is zero:

\begin{Shaded}
\begin{Highlighting}[]
\NormalTok{residuals <-}\StringTok{ }\KeywordTok{sum}\NormalTok{(}\KeywordTok{resid}\NormalTok{(modall))}
\end{Highlighting}
\end{Shaded}

The sum of the observed data is equal to the sum of the fitted values

\begin{Shaded}
\begin{Highlighting}[]
\NormalTok{Y_hat <-}\StringTok{ }\KeywordTok{predict}\NormalTok{(modall, bodyfat[}\DecValTok{1}\OperatorTok{:}\KeywordTok{length}\NormalTok{(}\KeywordTok{names}\NormalTok{(bodyfat))}\OperatorTok{-}\DecValTok{1}\NormalTok{])}
\KeywordTok{sum}\NormalTok{(bodyfat}\OperatorTok{$}\NormalTok{hwfat) }\OperatorTok{-}\StringTok{ }\KeywordTok{sum}\NormalTok{(Y_hat)}
\end{Highlighting}
\end{Shaded}

\begin{verbatim}
## [1] 4.547474e-13
\end{verbatim}

The residuals are orthogonal to the predictors

\begin{Shaded}
\begin{Highlighting}[]
\KeywordTok{sum}\NormalTok{(residuals}\OperatorTok{*}\NormalTok{bodyfat[}\DecValTok{1}\OperatorTok{:}\KeywordTok{length}\NormalTok{(}\KeywordTok{names}\NormalTok{(bodyfat))}\OperatorTok{-}\DecValTok{1}\NormalTok{])}
\end{Highlighting}
\end{Shaded}

\begin{verbatim}
## [1] -3.077268e-10
\end{verbatim}

The residuals are orthogonal to the fitted values

\begin{Shaded}
\begin{Highlighting}[]
\KeywordTok{sum}\NormalTok{(residuals}\OperatorTok{*}\NormalTok{Y_hat)}
\end{Highlighting}
\end{Shaded}

\begin{verbatim}
## [1] -1.568657e-11
\end{verbatim}

\hypertarget{exercise-6}{%
\section{Exercise 6}\label{exercise-6}}

\begin{Shaded}
\begin{Highlighting}[]
\CommentTok{# rsq <- function(x,y) cor(x,y)^2}
\CommentTok{# cols <- names(bodyfat)[1:length(names(bodyfat))-1]}
\CommentTok{# r_2 <- c()}
\CommentTok{# names(r_2) <- cols}
\CommentTok{# for (i in 1:length(cols)) \{}
\CommentTok{#     modall <- lm(hwfat ~ cols[i], bodyfat)}
\CommentTok{#     r_2 <- rsq(predict(hwfat))}
\CommentTok{# \}}
\CommentTok{# r2}
\end{Highlighting}
\end{Shaded}

\begin{Shaded}
\begin{Highlighting}[]
\KeywordTok{options}\NormalTok{(}\DataTypeTok{na.action =} \StringTok{"na.fail"}\NormalTok{)}
\NormalTok{modall <-}\StringTok{ }\KeywordTok{lm}\NormalTok{(hwfat }\OperatorTok{~}\NormalTok{., }\DataTypeTok{data =}\NormalTok{ bodyfat)}
\NormalTok{combs <-}\StringTok{ }\KeywordTok{dredge}\NormalTok{(modall, }\DataTypeTok{extra =} \StringTok{"R^2"}\NormalTok{)}
\end{Highlighting}
\end{Shaded}

\begin{verbatim}
## Fixed term is "(Intercept)"
\end{verbatim}

\begin{Shaded}
\begin{Highlighting}[]
\KeywordTok{print}\NormalTok{(}\StringTok{"best model"}\NormalTok{)}
\end{Highlighting}
\end{Shaded}

\begin{verbatim}
## [1] "best model"
\end{verbatim}

\begin{Shaded}
\begin{Highlighting}[]
\NormalTok{combs[combs}\OperatorTok{$}\StringTok{"R^2"} \OperatorTok{==}\StringTok{ }\KeywordTok{max}\NormalTok{(combs}\OperatorTok{$}\StringTok{"R^2"}\NormalTok{)]}
\end{Highlighting}
\end{Shaded}

\begin{verbatim}
## Global model call: lm(formula = hwfat ~ ., data = bodyfat)
## ---
## Model selection table 
##    (Intrc)    abs     age       ht  sbscp  trcps       wt    R^2 df  logLik
## 64   13.29 0.3714 -0.3289 -0.06731 0.1141 0.3874 -0.01365 0.8918  8 -193.43
##     AICc delta weight
## 64 404.9  5.58      1
## Models ranked by AICc(x)
\end{verbatim}

\end{document}
