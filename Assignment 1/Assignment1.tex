\PassOptionsToPackage{unicode=true}{hyperref} % options for packages loaded elsewhere
\PassOptionsToPackage{hyphens}{url}
%
\documentclass[]{article}
\usepackage{lmodern}
\usepackage{amssymb,amsmath}
\usepackage{ifxetex,ifluatex}
\usepackage{fixltx2e} % provides \textsubscript
\ifnum 0\ifxetex 1\fi\ifluatex 1\fi=0 % if pdftex
  \usepackage[T1]{fontenc}
  \usepackage[utf8]{inputenc}
  \usepackage{textcomp} % provides euro and other symbols
\else % if luatex or xelatex
  \usepackage{unicode-math}
  \defaultfontfeatures{Ligatures=TeX,Scale=MatchLowercase}
\fi
% use upquote if available, for straight quotes in verbatim environments
\IfFileExists{upquote.sty}{\usepackage{upquote}}{}
% use microtype if available
\IfFileExists{microtype.sty}{%
\usepackage[]{microtype}
\UseMicrotypeSet[protrusion]{basicmath} % disable protrusion for tt fonts
}{}
\IfFileExists{parskip.sty}{%
\usepackage{parskip}
}{% else
\setlength{\parindent}{0pt}
\setlength{\parskip}{6pt plus 2pt minus 1pt}
}
\usepackage{hyperref}
\hypersetup{
            pdftitle={Regression Models: Assignment 1},
            pdfauthor={Daniel Alonso},
            pdfborder={0 0 0},
            breaklinks=true}
\urlstyle{same}  % don't use monospace font for urls
\usepackage[margin=1in]{geometry}
\usepackage{color}
\usepackage{fancyvrb}
\newcommand{\VerbBar}{|}
\newcommand{\VERB}{\Verb[commandchars=\\\{\}]}
\DefineVerbatimEnvironment{Highlighting}{Verbatim}{commandchars=\\\{\}}
% Add ',fontsize=\small' for more characters per line
\usepackage{framed}
\definecolor{shadecolor}{RGB}{248,248,248}
\newenvironment{Shaded}{\begin{snugshade}}{\end{snugshade}}
\newcommand{\AlertTok}[1]{\textcolor[rgb]{0.94,0.16,0.16}{#1}}
\newcommand{\AnnotationTok}[1]{\textcolor[rgb]{0.56,0.35,0.01}{\textbf{\textit{#1}}}}
\newcommand{\AttributeTok}[1]{\textcolor[rgb]{0.77,0.63,0.00}{#1}}
\newcommand{\BaseNTok}[1]{\textcolor[rgb]{0.00,0.00,0.81}{#1}}
\newcommand{\BuiltInTok}[1]{#1}
\newcommand{\CharTok}[1]{\textcolor[rgb]{0.31,0.60,0.02}{#1}}
\newcommand{\CommentTok}[1]{\textcolor[rgb]{0.56,0.35,0.01}{\textit{#1}}}
\newcommand{\CommentVarTok}[1]{\textcolor[rgb]{0.56,0.35,0.01}{\textbf{\textit{#1}}}}
\newcommand{\ConstantTok}[1]{\textcolor[rgb]{0.00,0.00,0.00}{#1}}
\newcommand{\ControlFlowTok}[1]{\textcolor[rgb]{0.13,0.29,0.53}{\textbf{#1}}}
\newcommand{\DataTypeTok}[1]{\textcolor[rgb]{0.13,0.29,0.53}{#1}}
\newcommand{\DecValTok}[1]{\textcolor[rgb]{0.00,0.00,0.81}{#1}}
\newcommand{\DocumentationTok}[1]{\textcolor[rgb]{0.56,0.35,0.01}{\textbf{\textit{#1}}}}
\newcommand{\ErrorTok}[1]{\textcolor[rgb]{0.64,0.00,0.00}{\textbf{#1}}}
\newcommand{\ExtensionTok}[1]{#1}
\newcommand{\FloatTok}[1]{\textcolor[rgb]{0.00,0.00,0.81}{#1}}
\newcommand{\FunctionTok}[1]{\textcolor[rgb]{0.00,0.00,0.00}{#1}}
\newcommand{\ImportTok}[1]{#1}
\newcommand{\InformationTok}[1]{\textcolor[rgb]{0.56,0.35,0.01}{\textbf{\textit{#1}}}}
\newcommand{\KeywordTok}[1]{\textcolor[rgb]{0.13,0.29,0.53}{\textbf{#1}}}
\newcommand{\NormalTok}[1]{#1}
\newcommand{\OperatorTok}[1]{\textcolor[rgb]{0.81,0.36,0.00}{\textbf{#1}}}
\newcommand{\OtherTok}[1]{\textcolor[rgb]{0.56,0.35,0.01}{#1}}
\newcommand{\PreprocessorTok}[1]{\textcolor[rgb]{0.56,0.35,0.01}{\textit{#1}}}
\newcommand{\RegionMarkerTok}[1]{#1}
\newcommand{\SpecialCharTok}[1]{\textcolor[rgb]{0.00,0.00,0.00}{#1}}
\newcommand{\SpecialStringTok}[1]{\textcolor[rgb]{0.31,0.60,0.02}{#1}}
\newcommand{\StringTok}[1]{\textcolor[rgb]{0.31,0.60,0.02}{#1}}
\newcommand{\VariableTok}[1]{\textcolor[rgb]{0.00,0.00,0.00}{#1}}
\newcommand{\VerbatimStringTok}[1]{\textcolor[rgb]{0.31,0.60,0.02}{#1}}
\newcommand{\WarningTok}[1]{\textcolor[rgb]{0.56,0.35,0.01}{\textbf{\textit{#1}}}}
\usepackage{graphicx,grffile}
\makeatletter
\def\maxwidth{\ifdim\Gin@nat@width>\linewidth\linewidth\else\Gin@nat@width\fi}
\def\maxheight{\ifdim\Gin@nat@height>\textheight\textheight\else\Gin@nat@height\fi}
\makeatother
% Scale images if necessary, so that they will not overflow the page
% margins by default, and it is still possible to overwrite the defaults
% using explicit options in \includegraphics[width, height, ...]{}
\setkeys{Gin}{width=\maxwidth,height=\maxheight,keepaspectratio}
\setlength{\emergencystretch}{3em}  % prevent overfull lines
\providecommand{\tightlist}{%
  \setlength{\itemsep}{0pt}\setlength{\parskip}{0pt}}
\setcounter{secnumdepth}{0}
% Redefines (sub)paragraphs to behave more like sections
\ifx\paragraph\undefined\else
\let\oldparagraph\paragraph
\renewcommand{\paragraph}[1]{\oldparagraph{#1}\mbox{}}
\fi
\ifx\subparagraph\undefined\else
\let\oldsubparagraph\subparagraph
\renewcommand{\subparagraph}[1]{\oldsubparagraph{#1}\mbox{}}
\fi

% set default figure placement to htbp
\makeatletter
\def\fps@figure{htbp}
\makeatother


\title{Regression Models: Assignment 1}
\author{Daniel Alonso}
\date{November 24th, 2020}

\begin{document}
\maketitle

Importing libraries

\begin{Shaded}
\begin{Highlighting}[]
\KeywordTok{library}\NormalTok{(dplyr)}
\KeywordTok{library}\NormalTok{(MuMIn)}
\KeywordTok{library}\NormalTok{(MASS)}
\end{Highlighting}
\end{Shaded}

\hypertarget{exercise-1}{%
\section{Exercise 1}\label{exercise-1}}

\hypertarget{simulation}{%
\subsection{Simulation}\label{simulation}}

\begin{Shaded}
\begin{Highlighting}[]
\NormalTok{sim =}\StringTok{ }\KeywordTok{list}\NormalTok{()}
\ControlFlowTok{for}\NormalTok{ (j }\ControlFlowTok{in} \DecValTok{1}\OperatorTok{:}\DecValTok{1000}\NormalTok{) \{}
\NormalTok{    vals =}\StringTok{ }\KeywordTok{c}\NormalTok{()}
    \ControlFlowTok{for}\NormalTok{ (i }\ControlFlowTok{in} \DecValTok{1}\OperatorTok{:}\DecValTok{100}\NormalTok{) \{}
\NormalTok{        run =}\StringTok{ }\DecValTok{3} \OperatorTok{+}\StringTok{ }\DecValTok{3}\OperatorTok{*}\KeywordTok{cos}\NormalTok{(i}\OperatorTok{/}\DecValTok{10} \OperatorTok{+}\StringTok{ }\DecValTok{50}\NormalTok{) }\OperatorTok{+}\StringTok{ }\KeywordTok{rnorm}\NormalTok{(}\DecValTok{1}\NormalTok{, }\DataTypeTok{mean=}\DecValTok{0}\NormalTok{, }\DataTypeTok{sd=}\DecValTok{1}\NormalTok{)}
\NormalTok{        vals =}\StringTok{ }\KeywordTok{c}\NormalTok{(vals, run)}
\NormalTok{    \}}
\NormalTok{    sim[[j]] =}\StringTok{ }\NormalTok{vals}
\NormalTok{\}}
\NormalTok{sim}
\end{Highlighting}
\end{Shaded}

\hypertarget{exercise-2}{%
\section{Exercise 2}\label{exercise-2}}

\hypertarget{importing-the-data}{%
\subsection{Importing the data}\label{importing-the-data}}

\begin{Shaded}
\begin{Highlighting}[]
\NormalTok{d <-}\StringTok{ }\KeywordTok{data.frame}\NormalTok{(}\KeywordTok{read.table}\NormalTok{(}\StringTok{'../data/index.txt'}\NormalTok{, }\DataTypeTok{header=}\OtherTok{TRUE}\NormalTok{))}
\end{Highlighting}
\end{Shaded}

\begin{Shaded}
\begin{Highlighting}[]
\NormalTok{X =}\StringTok{ }\NormalTok{d}\OperatorTok{$}\NormalTok{PovPct}
\NormalTok{Y =}\StringTok{ }\NormalTok{d}\OperatorTok{$}\NormalTok{Brth15to17}
\NormalTok{beta1 =}\StringTok{ }\KeywordTok{cov}\NormalTok{(X, Y)}\OperatorTok{/}\KeywordTok{var}\NormalTok{(X)}
\NormalTok{beta0 =}\StringTok{ }\KeywordTok{mean}\NormalTok{(Y) }\OperatorTok{-}\StringTok{ }\NormalTok{beta1}\OperatorTok{*}\KeywordTok{mean}\NormalTok{(X)}
\end{Highlighting}
\end{Shaded}

\begin{Shaded}
\begin{Highlighting}[]
\NormalTok{beta1}
\end{Highlighting}
\end{Shaded}

\begin{verbatim}
## [1] 1.373345
\end{verbatim}

\begin{Shaded}
\begin{Highlighting}[]
\NormalTok{beta0}
\end{Highlighting}
\end{Shaded}

\begin{verbatim}
## [1] 4.267293
\end{verbatim}

\newpage

\hypertarget{exercise-3}{%
\section{Exercise 3}\label{exercise-3}}

First we have the log-likelihood function for \(\beta\) and
\(\sigma^{2}\)

\(l(\sigma^{2} | X) = \sum_{i=1}^n log(\frac{1}{\sqrt{2 \pi \sigma^{2}}} - \frac{(Y_{i} - (\beta_{0} + \beta_{1} x_{ik} + \dots + \beta_{k} x_{ik}))^{2}}{2 \sigma^{2}})\)

\(\propto - \frac{n}{2} log(\sigma^{2}) - \frac{(Y - X \beta) \prime (Y - X \beta)}{2 \sigma^{2}}\)

Differentiating the second expression:

\(\frac{\partial l}{\partial \sigma} ( - \frac{n}{2} log(\sigma^{2}) - \frac{(Y - X \beta) \prime (Y - X \beta)}{2 \sigma^{2}}) = 0\)

We get:

\(- \frac{n}{2} (\frac{1}{ \sigma^{2}} ) (2 \sigma) - (Y - X \beta) \prime (Y - X \beta) * (-2)(2 \sigma^{-3}) = 0\)

We reduce the expression further:

\(- \frac{n}{\sigma} + \frac{(Y - X \beta) \prime (Y - X \beta)}{\sigma^{3}} = 0\)

We multiply both sides by \(\sigma^{3}\) and we get:

\(- n \sigma^{2} + (Y - X \beta) \prime (Y - X \beta) = 0\)

And solving for \(\sigma^{2}\) we get:

\(\hat{\sigma^{2}} = \frac{(Y - X \beta) \prime (Y - X \beta)}{n}\)

Which is our maximum likelihood estimator for \(\sigma^2\)

\hypertarget{exercise-4}{%
\section{Exercise 4}\label{exercise-4}}

\newpage

\hypertarget{exercise-5}{%
\section{Exercise 5}\label{exercise-5}}

\begin{Shaded}
\begin{Highlighting}[]
\NormalTok{bodyfat <-}\StringTok{ }\KeywordTok{data.frame}\NormalTok{(}\KeywordTok{read.table}\NormalTok{(}\StringTok{'../data/bodyfat.txt'}\NormalTok{, }\DataTypeTok{header=}\OtherTok{TRUE}\NormalTok{))}
\NormalTok{modall <-}\StringTok{ }\KeywordTok{lm}\NormalTok{(hwfat }\OperatorTok{~}\NormalTok{., }\DataTypeTok{data =}\NormalTok{ bodyfat)}
\KeywordTok{summary}\NormalTok{(modall)}
\end{Highlighting}
\end{Shaded}

\begin{verbatim}
## 
## Call:
## lm(formula = hwfat ~ ., data = bodyfat)
## 
## Residuals:
##    Min     1Q Median     3Q    Max 
## -6.162 -1.858 -0.464  2.502  8.177 
## 
## Coefficients:
##             Estimate Std. Error t value Pr(>|t|)    
## (Intercept) 13.29370    9.63027   1.380   0.1718    
## age         -0.32893    0.32158  -1.023   0.3098    
## ht          -0.06731    0.16051  -0.419   0.6762    
## wt          -0.01365    0.02591  -0.527   0.5999    
## abs          0.37142    0.08837   4.203 7.55e-05 ***
## triceps      0.38743    0.13761   2.815   0.0063 ** 
## subscap      0.11405    0.14193   0.804   0.4243    
## ---
## Signif. codes:  0 '***' 0.001 '**' 0.01 '*' 0.05 '.' 0.1 ' ' 1
## 
## Residual standard error: 3.028 on 71 degrees of freedom
## Multiple R-squared:  0.8918, Adjusted R-squared:  0.8827 
## F-statistic: 97.54 on 6 and 71 DF,  p-value: < 2.2e-16
\end{verbatim}

The sum of residuals is zero:

\begin{Shaded}
\begin{Highlighting}[]
\NormalTok{residuals <-}\StringTok{ }\KeywordTok{sum}\NormalTok{(}\KeywordTok{resid}\NormalTok{(modall))}
\end{Highlighting}
\end{Shaded}

The sum of the observed data is equal to the sum of the fitted values

\begin{Shaded}
\begin{Highlighting}[]
\NormalTok{Y_hat <-}\StringTok{ }\KeywordTok{predict}\NormalTok{(modall, bodyfat[}\DecValTok{1}\OperatorTok{:}\KeywordTok{length}\NormalTok{(}\KeywordTok{names}\NormalTok{(bodyfat))}\OperatorTok{-}\DecValTok{1}\NormalTok{])}
\KeywordTok{sum}\NormalTok{(bodyfat}\OperatorTok{$}\NormalTok{hwfat) }\OperatorTok{-}\StringTok{ }\KeywordTok{sum}\NormalTok{(Y_hat)}
\end{Highlighting}
\end{Shaded}

\begin{verbatim}
## [1] 4.547474e-13
\end{verbatim}

The residuals are orthogonal to the predictors

\begin{Shaded}
\begin{Highlighting}[]
\KeywordTok{sum}\NormalTok{(residuals}\OperatorTok{*}\NormalTok{bodyfat[}\DecValTok{1}\OperatorTok{:}\KeywordTok{length}\NormalTok{(}\KeywordTok{names}\NormalTok{(bodyfat))}\OperatorTok{-}\DecValTok{1}\NormalTok{])}
\end{Highlighting}
\end{Shaded}

\begin{verbatim}
## [1] -3.077268e-10
\end{verbatim}

The residuals are orthogonal to the fitted values

\begin{Shaded}
\begin{Highlighting}[]
\KeywordTok{sum}\NormalTok{(residuals}\OperatorTok{*}\NormalTok{Y_hat)}
\end{Highlighting}
\end{Shaded}

\begin{verbatim}
## [1] -1.568657e-11
\end{verbatim}

\newpage

\hypertarget{exercise-6}{%
\section{Exercise 6}\label{exercise-6}}

\begin{Shaded}
\begin{Highlighting}[]
\KeywordTok{options}\NormalTok{(}\DataTypeTok{na.action =} \StringTok{"na.fail"}\NormalTok{)}
\NormalTok{modall <-}\StringTok{ }\KeywordTok{lm}\NormalTok{(hwfat }\OperatorTok{~}\NormalTok{., }\DataTypeTok{data =}\NormalTok{ bodyfat)}
\NormalTok{combs <-}\StringTok{ }\KeywordTok{dredge}\NormalTok{(modall, }\DataTypeTok{extra =} \StringTok{"R^2"}\NormalTok{)}
\end{Highlighting}
\end{Shaded}

\begin{verbatim}
## Fixed term is "(Intercept)"
\end{verbatim}

\begin{Shaded}
\begin{Highlighting}[]
\KeywordTok{print}\NormalTok{(}\StringTok{"best model"}\NormalTok{)}
\end{Highlighting}
\end{Shaded}

\begin{verbatim}
## [1] "best model"
\end{verbatim}

\begin{Shaded}
\begin{Highlighting}[]
\NormalTok{combs[combs}\OperatorTok{$}\StringTok{"R^2"} \OperatorTok{==}\StringTok{ }\KeywordTok{max}\NormalTok{(combs}\OperatorTok{$}\StringTok{"R^2"}\NormalTok{)]}
\end{Highlighting}
\end{Shaded}

\begin{verbatim}
## Global model call: lm(formula = hwfat ~ ., data = bodyfat)
## ---
## Model selection table 
##    (Intrc)    abs     age       ht  sbscp  trcps       wt    R^2 df  logLik
## 64   13.29 0.3714 -0.3289 -0.06731 0.1141 0.3874 -0.01365 0.8918  8 -193.43
##     AICc delta weight
## 64 404.9  5.58      1
## Models ranked by AICc(x)
\end{verbatim}

\hypertarget{exercise-7}{%
\section{Exercise 7}\label{exercise-7}}

\newpage

\hypertarget{exercise-8}{%
\section{Exercise 8}\label{exercise-8}}

We define a list with all the models excluding, in each one, a single
variable.

\begin{Shaded}
\begin{Highlighting}[]
\NormalTok{models <-}\StringTok{ }\KeywordTok{list}\NormalTok{()}
\NormalTok{vars <-}\StringTok{ }\KeywordTok{c}\NormalTok{(}\StringTok{"age"}\NormalTok{,}\StringTok{"ht"}\NormalTok{,}\StringTok{"wt"}\NormalTok{,}\StringTok{"abs"}\NormalTok{,}\StringTok{"triceps"}\NormalTok{,}\StringTok{"subscap"}\NormalTok{)}
\NormalTok{models[[}\DecValTok{1}\NormalTok{]] <-}\StringTok{ }\KeywordTok{update}\NormalTok{(modall,.}\OperatorTok{~}\NormalTok{.}\OperatorTok{-}\NormalTok{age)}
\NormalTok{models[[}\DecValTok{2}\NormalTok{]] <-}\StringTok{ }\KeywordTok{update}\NormalTok{(modall,.}\OperatorTok{~}\NormalTok{.}\OperatorTok{-}\NormalTok{ht)}
\NormalTok{models[[}\DecValTok{3}\NormalTok{]] <-}\StringTok{ }\KeywordTok{update}\NormalTok{(modall,.}\OperatorTok{~}\NormalTok{.}\OperatorTok{-}\NormalTok{wt)}
\NormalTok{models[[}\DecValTok{4}\NormalTok{]] <-}\StringTok{ }\KeywordTok{update}\NormalTok{(modall,.}\OperatorTok{~}\NormalTok{.}\OperatorTok{-}\NormalTok{abs)}
\NormalTok{models[[}\DecValTok{5}\NormalTok{]] <-}\StringTok{ }\KeywordTok{update}\NormalTok{(modall,.}\OperatorTok{~}\NormalTok{.}\OperatorTok{-}\NormalTok{triceps)}
\NormalTok{models[[}\DecValTok{6}\NormalTok{]] <-}\StringTok{ }\KeywordTok{update}\NormalTok{(modall,.}\OperatorTok{~}\NormalTok{.}\OperatorTok{-}\NormalTok{subscap)}
\end{Highlighting}
\end{Shaded}

We run ANOVA with both the models without each variable and the main
model including all the other variables.

We can see the pvalues for the ANOVA where each specific variable was
excluded:

\begin{Shaded}
\begin{Highlighting}[]
\NormalTok{anovas <-}\StringTok{ }\KeywordTok{list}\NormalTok{()}
\NormalTok{pvalues <-}\StringTok{ }\KeywordTok{c}\NormalTok{()}
\NormalTok{amount_of_vars <-}\StringTok{ }\KeywordTok{length}\NormalTok{(}\KeywordTok{names}\NormalTok{(bodyfat))}\OperatorTok{-}\DecValTok{1}
\ControlFlowTok{for}\NormalTok{ (i }\ControlFlowTok{in} \DecValTok{1}\OperatorTok{:}\NormalTok{amount_of_vars) \{}
\NormalTok{    anovas[[i]] <-}\StringTok{ }\KeywordTok{anova}\NormalTok{(models[[i]],modall)}
\NormalTok{    pvalues <-}\StringTok{ }\KeywordTok{c}\NormalTok{(pvalues, }\KeywordTok{sum}\NormalTok{(anovas[[i]][}\DecValTok{2}\NormalTok{,}\StringTok{"Pr(>F)"}\NormalTok{]))}
\NormalTok{\}}
\ControlFlowTok{for}\NormalTok{ (i }\ControlFlowTok{in} \DecValTok{1}\OperatorTok{:}\KeywordTok{length}\NormalTok{(vars)) \{}
    \KeywordTok{print}\NormalTok{(}\KeywordTok{paste}\NormalTok{(}\StringTok{"excluding: "}\NormalTok{, vars[i], }\StringTok{": "}\NormalTok{, pvalues[i] , }\DataTypeTok{sep=}\StringTok{""}\NormalTok{))}
\NormalTok{\}}
\end{Highlighting}
\end{Shaded}

\begin{verbatim}
## [1] "excluding: age: 0.30983932449522"
## [1] "excluding: ht: 0.67622546378066"
## [1] "excluding: wt: 0.599878887504826"
## [1] "excluding: abs: 7.54898491342447e-05"
## [1] "excluding: triceps: 0.00630111253287972"
## [1] "excluding: subscap: 0.424314507846979"
\end{verbatim}

Then we compare with summary:

\begin{Shaded}
\begin{Highlighting}[]
\KeywordTok{summary}\NormalTok{(modall)[}\DecValTok{4}\NormalTok{]}
\end{Highlighting}
\end{Shaded}

\begin{verbatim}
## $coefficients
##                Estimate Std. Error    t value     Pr(>|t|)
## (Intercept) 13.29369860 9.63026704  1.3804081 1.717917e-01
## age         -0.32893403 0.32157778 -1.0228755 3.098393e-01
## ht          -0.06730905 0.16050751 -0.4193514 6.762255e-01
## wt          -0.01365183 0.02590783 -0.5269385 5.998789e-01
## abs          0.37141976 0.08836595  4.2032001 7.548985e-05
## triceps      0.38742647 0.13761017  2.8153912 6.301113e-03
## subscap      0.11405213 0.14192779  0.8035927 4.243145e-01
\end{verbatim}

And we can see we get the same pvalues in the summary. Therefore viewing
the summary can be a much faster version of performing such testing.

as a result we get that the least meaningful variable (the variable that
explains the lowest variance of the model) is the variable ht (height)
followed by the variable wt (weight).

\newpage

\hypertarget{exercise-9}{%
\section{Exercise 9}\label{exercise-9}}

Given that
\(E[\hat{Y} | X_{h}] = \hat{Y_{h}} \sim N(X_{h} \beta, \sigma^{2} X_{h} (X^{\prime} X) X_{h}^{\prime})\)

\(\Rightarrow \hat{y_{h}} \pm t_{n - (k+1), \frac{\alpha}{2}} * \hat{\sigma} \sqrt{h_{ii}}\)

where \(h_{ii}\) is the diagonal of our \(H\) matrix.

is our expression for the \((1-\alpha)\%\) confidence interval for
\(\hat{Y_{h}}\) when \(\sigma^{2}\) is unknown.

\hypertarget{exercise-10}{%
\section{Exercise 10}\label{exercise-10}}

\begin{Shaded}
\begin{Highlighting}[]
\NormalTok{transform <-}\StringTok{ }\KeywordTok{data.frame}\NormalTok{(}\KeywordTok{read.table}\NormalTok{(}\StringTok{'../data/Transform2_V2.txt'}\NormalTok{, }\DataTypeTok{header=}\OtherTok{TRUE}\NormalTok{))}
\end{Highlighting}
\end{Shaded}

\end{document}
