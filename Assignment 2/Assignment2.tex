\PassOptionsToPackage{unicode=true}{hyperref} % options for packages loaded elsewhere
\PassOptionsToPackage{hyphens}{url}
%
\documentclass[]{article}
\usepackage{lmodern}
\usepackage{amssymb,amsmath}
\usepackage{ifxetex,ifluatex}
\usepackage{fixltx2e} % provides \textsubscript
\ifnum 0\ifxetex 1\fi\ifluatex 1\fi=0 % if pdftex
  \usepackage[T1]{fontenc}
  \usepackage[utf8]{inputenc}
  \usepackage{textcomp} % provides euro and other symbols
\else % if luatex or xelatex
  \usepackage{unicode-math}
  \defaultfontfeatures{Ligatures=TeX,Scale=MatchLowercase}
\fi
% use upquote if available, for straight quotes in verbatim environments
\IfFileExists{upquote.sty}{\usepackage{upquote}}{}
% use microtype if available
\IfFileExists{microtype.sty}{%
\usepackage[]{microtype}
\UseMicrotypeSet[protrusion]{basicmath} % disable protrusion for tt fonts
}{}
\IfFileExists{parskip.sty}{%
\usepackage{parskip}
}{% else
\setlength{\parindent}{0pt}
\setlength{\parskip}{6pt plus 2pt minus 1pt}
}
\usepackage{hyperref}
\hypersetup{
            pdftitle={Regression Models: Assignment 2},
            pdfauthor={Daniel Alonso},
            pdfborder={0 0 0},
            breaklinks=true}
\urlstyle{same}  % don't use monospace font for urls
\usepackage[margin=1in]{geometry}
\usepackage{color}
\usepackage{fancyvrb}
\newcommand{\VerbBar}{|}
\newcommand{\VERB}{\Verb[commandchars=\\\{\}]}
\DefineVerbatimEnvironment{Highlighting}{Verbatim}{commandchars=\\\{\}}
% Add ',fontsize=\small' for more characters per line
\usepackage{framed}
\definecolor{shadecolor}{RGB}{248,248,248}
\newenvironment{Shaded}{\begin{snugshade}}{\end{snugshade}}
\newcommand{\AlertTok}[1]{\textcolor[rgb]{0.94,0.16,0.16}{#1}}
\newcommand{\AnnotationTok}[1]{\textcolor[rgb]{0.56,0.35,0.01}{\textbf{\textit{#1}}}}
\newcommand{\AttributeTok}[1]{\textcolor[rgb]{0.77,0.63,0.00}{#1}}
\newcommand{\BaseNTok}[1]{\textcolor[rgb]{0.00,0.00,0.81}{#1}}
\newcommand{\BuiltInTok}[1]{#1}
\newcommand{\CharTok}[1]{\textcolor[rgb]{0.31,0.60,0.02}{#1}}
\newcommand{\CommentTok}[1]{\textcolor[rgb]{0.56,0.35,0.01}{\textit{#1}}}
\newcommand{\CommentVarTok}[1]{\textcolor[rgb]{0.56,0.35,0.01}{\textbf{\textit{#1}}}}
\newcommand{\ConstantTok}[1]{\textcolor[rgb]{0.00,0.00,0.00}{#1}}
\newcommand{\ControlFlowTok}[1]{\textcolor[rgb]{0.13,0.29,0.53}{\textbf{#1}}}
\newcommand{\DataTypeTok}[1]{\textcolor[rgb]{0.13,0.29,0.53}{#1}}
\newcommand{\DecValTok}[1]{\textcolor[rgb]{0.00,0.00,0.81}{#1}}
\newcommand{\DocumentationTok}[1]{\textcolor[rgb]{0.56,0.35,0.01}{\textbf{\textit{#1}}}}
\newcommand{\ErrorTok}[1]{\textcolor[rgb]{0.64,0.00,0.00}{\textbf{#1}}}
\newcommand{\ExtensionTok}[1]{#1}
\newcommand{\FloatTok}[1]{\textcolor[rgb]{0.00,0.00,0.81}{#1}}
\newcommand{\FunctionTok}[1]{\textcolor[rgb]{0.00,0.00,0.00}{#1}}
\newcommand{\ImportTok}[1]{#1}
\newcommand{\InformationTok}[1]{\textcolor[rgb]{0.56,0.35,0.01}{\textbf{\textit{#1}}}}
\newcommand{\KeywordTok}[1]{\textcolor[rgb]{0.13,0.29,0.53}{\textbf{#1}}}
\newcommand{\NormalTok}[1]{#1}
\newcommand{\OperatorTok}[1]{\textcolor[rgb]{0.81,0.36,0.00}{\textbf{#1}}}
\newcommand{\OtherTok}[1]{\textcolor[rgb]{0.56,0.35,0.01}{#1}}
\newcommand{\PreprocessorTok}[1]{\textcolor[rgb]{0.56,0.35,0.01}{\textit{#1}}}
\newcommand{\RegionMarkerTok}[1]{#1}
\newcommand{\SpecialCharTok}[1]{\textcolor[rgb]{0.00,0.00,0.00}{#1}}
\newcommand{\SpecialStringTok}[1]{\textcolor[rgb]{0.31,0.60,0.02}{#1}}
\newcommand{\StringTok}[1]{\textcolor[rgb]{0.31,0.60,0.02}{#1}}
\newcommand{\VariableTok}[1]{\textcolor[rgb]{0.00,0.00,0.00}{#1}}
\newcommand{\VerbatimStringTok}[1]{\textcolor[rgb]{0.31,0.60,0.02}{#1}}
\newcommand{\WarningTok}[1]{\textcolor[rgb]{0.56,0.35,0.01}{\textbf{\textit{#1}}}}
\usepackage{graphicx,grffile}
\makeatletter
\def\maxwidth{\ifdim\Gin@nat@width>\linewidth\linewidth\else\Gin@nat@width\fi}
\def\maxheight{\ifdim\Gin@nat@height>\textheight\textheight\else\Gin@nat@height\fi}
\makeatother
% Scale images if necessary, so that they will not overflow the page
% margins by default, and it is still possible to overwrite the defaults
% using explicit options in \includegraphics[width, height, ...]{}
\setkeys{Gin}{width=\maxwidth,height=\maxheight,keepaspectratio}
\setlength{\emergencystretch}{3em}  % prevent overfull lines
\providecommand{\tightlist}{%
  \setlength{\itemsep}{0pt}\setlength{\parskip}{0pt}}
\setcounter{secnumdepth}{0}
% Redefines (sub)paragraphs to behave more like sections
\ifx\paragraph\undefined\else
\let\oldparagraph\paragraph
\renewcommand{\paragraph}[1]{\oldparagraph{#1}\mbox{}}
\fi
\ifx\subparagraph\undefined\else
\let\oldsubparagraph\subparagraph
\renewcommand{\subparagraph}[1]{\oldsubparagraph{#1}\mbox{}}
\fi

% set default figure placement to htbp
\makeatletter
\def\fps@figure{htbp}
\makeatother


\title{Regression Models: Assignment 2}
\author{Daniel Alonso}
\date{January 11th, 2020}

\begin{document}
\maketitle

\hypertarget{installing-libraries-used}{%
\subsection{Installing libraries used}\label{installing-libraries-used}}

\begin{Shaded}
\begin{Highlighting}[]
\NormalTok{packages =}\StringTok{ }\KeywordTok{c}\NormalTok{(}\StringTok{"dplyr"}\NormalTok{,}\StringTok{"MuMIn"}\NormalTok{,}\StringTok{"MASS"}\NormalTok{,}\StringTok{"leaps"}\NormalTok{,}\StringTok{"glmnet"}\NormalTok{,}\StringTok{"car"}\NormalTok{)}
\ControlFlowTok{for}\NormalTok{ (package }\ControlFlowTok{in}\NormalTok{ packages) \{}
    \KeywordTok{install.packages}\NormalTok{(package)}
\NormalTok{\}}
\end{Highlighting}
\end{Shaded}

\hypertarget{importing-libraries}{%
\subsection{Importing libraries}\label{importing-libraries}}

\begin{Shaded}
\begin{Highlighting}[]
\KeywordTok{library}\NormalTok{(dplyr)}
\KeywordTok{library}\NormalTok{(MuMIn)}
\KeywordTok{library}\NormalTok{(MASS)}
\KeywordTok{library}\NormalTok{(leaps)}
\KeywordTok{library}\NormalTok{(glmnet)}
\KeywordTok{library}\NormalTok{(car)}
\end{Highlighting}
\end{Shaded}

\newpage

\hypertarget{exercise-1}{%
\subsection{Exercise 1}\label{exercise-1}}

\hypertarget{model-and-parameter-interpretation}{%
\subsubsection{1 - Model and parameter
interpretation}\label{model-and-parameter-interpretation}}

\emph{Y} = Binary variable representing whether the customer will buy a
car or not \emph{income} = annual family income

Therefore we model the response as:

\(Y = \beta_{0} + \beta_{1} X + \epsilon\)

And our values will be:

\(Y = -1.98079 + 0.04342 X + \epsilon\)

The higher the family income, the higher the likeliness that a family
will purchase a new car.

\begin{Shaded}
\begin{Highlighting}[]
\NormalTok{y =}\StringTok{ }\ControlFlowTok{function}\NormalTok{(x) (}\OperatorTok{-}\FloatTok{1.98079} \OperatorTok{+}\StringTok{ }\FloatTok{0.04342} \OperatorTok{*}\StringTok{ }\NormalTok{x)}
\NormalTok{p =}\StringTok{ }\ControlFlowTok{function}\NormalTok{(y) (}\KeywordTok{exp}\NormalTok{(y)}\OperatorTok{/}\NormalTok{(}\KeywordTok{exp}\NormalTok{(y)}\OperatorTok{+}\DecValTok{1}\NormalTok{))}
\end{Highlighting}
\end{Shaded}

\hypertarget{ci-for-the-probability-that-a-family-with-annual-income-of-60-thousand-dollars-will-purchase-anew-car-next-year.}{%
\subsubsection{2 - 95\%-CI for the probability that a family with annual
income of 60 thousand dollars will purchase anew car next
year.}\label{ci-for-the-probability-that-a-family-with-annual-income-of-60-thousand-dollars-will-purchase-anew-car-next-year.}}

We calculate the asymptotic \((1 - \alpha)\%\) confidence interval:

\(\hat{\beta_{j}} \pm z_{\frac{\alpha}{2}} S.E. (\hat{\beta_{j}})\)

With our values we get:

\begin{Shaded}
\begin{Highlighting}[]
\CommentTok{# CI calculation}
\NormalTok{l_bound <-}\StringTok{ }\FloatTok{0.04342} \OperatorTok{-}\StringTok{ }\KeywordTok{qnorm}\NormalTok{(}\FloatTok{0.975}\NormalTok{) }\OperatorTok{*}\StringTok{ }\FloatTok{0.02011}
\NormalTok{u_bound <-}\StringTok{ }\FloatTok{0.04342} \OperatorTok{+}\StringTok{ }\KeywordTok{qnorm}\NormalTok{(}\FloatTok{0.975}\NormalTok{) }\OperatorTok{*}\StringTok{ }\FloatTok{0.02011}
\end{Highlighting}
\end{Shaded}

\(0.004005124 \leq \hat{\beta_{j}} \leq 0.08283488\)

\begin{Shaded}
\begin{Highlighting}[]
\KeywordTok{p}\NormalTok{(l_bound)}
\CommentTok{#> [1] 0.5010013}
\KeywordTok{p}\NormalTok{(u_bound)}
\CommentTok{#> [1] 0.5206969}
\end{Highlighting}
\end{Shaded}

\(0.5010013 \leq p \leq 0.5206969\)

Therefore for 60 thousand dollar annual income households:

\begin{Shaded}
\begin{Highlighting}[]
\NormalTok{L <-}\StringTok{ }\FloatTok{-1.98079} \OperatorTok{+}\StringTok{ }\FloatTok{0.004005124} \OperatorTok{*}\StringTok{ }\DecValTok{60}
\NormalTok{U <-}\StringTok{ }\FloatTok{-1.98079} \OperatorTok{+}\StringTok{ }\FloatTok{0.08283488} \OperatorTok{*}\StringTok{ }\DecValTok{60}
\KeywordTok{p}\NormalTok{(L)}
\CommentTok{#> [1] 0.1492517}
\KeywordTok{p}\NormalTok{(U)}
\CommentTok{#> [1] 0.9520885}
\end{Highlighting}
\end{Shaded}

\(0.1492517 \leq p_{60k} \leq 0.9520885\)

\end{document}
